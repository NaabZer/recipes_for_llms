\documentclass[11pt]{article}
% Change "review" to "final" to generate the final (sometimes called camera-ready) version.
% Change to "preprint" to generate a non-anonymous version with page numbers.
\usepackage[review]{acl}

% Standard package includes
\usepackage{times}
\usepackage{latexsym}

% For proper rendering and hyphenation of words containing Latin characters (including in bib files)
\usepackage[T1]{fontenc}
% For Vietnamese characters
% \usepackage[T5]{fontenc}
% See https://www.latex-project.org/help/documentation/encguide.pdf for other character sets

% This assumes your files are encoded as UTF8
\usepackage[utf8]{inputenc}

% This is not strictly necessary, and may be commented out,
% but it will improve the layout of the manuscript,
% and will typically save some space.
\usepackage{microtype}

% This is also not strictly necessary, and may be commented out.
% However, it will improve the aesthetics of text in
% the typewriter font.
\usepackage{inconsolata}

%Including images in your LaTeX document requires adding
%additional package(s)
\usepackage{graphicx}

\usepackage{amsmath}
\usepackage{mathtools}
\usepackage[utf8]{inputenc}
\usepackage{hyperref}


\title{Natural Language Processing for efficient recipe retrieval}
\date{}
\author{Anton Gefvert \\ Linköping University}

\begin{document}
\maketitle
\begin{abstract}
    lorem ipsum
\end{abstract}

\section{Introduction}

\section{Theory}

\section{Data}

\section{Method}
\subsection{Baselines}
\begin{itemize}
    \item BOW (tfidf)
    \item BERT embeddings
\end{itemize}

\subsection{Better methods}
\begin{itemize}
    \item BOW using NER filtering
    \item NER based parquet + duckdb database with custom scoring function
    \item GraphDB database (this most likely also needs a scoring function)
\end{itemize}

\subsection{Evaluation Tasks}
To see if the developed models performs the task of recipe retrieval from
ingredients, some evaluation tasks are defined.
Two types of task is used in this paper, model performance and query speed.
The first one to evaluate if the developed method works well, specifically in
relation the the baselines.
The second one to evaluate the speed of which these queries are handled, to
gauge how suitable they are for different systems.
For example, in a load heavy system that require low response times, model
performance might be compromisable as long as the query speed remains low.

\subsubsection{Model Performance}\label{sec:modelperfomance}
To evaluate the model performance, four different evaluation methods were
created.
All the tasks are evaluated using \emph{Accuray},
\emph{Mean Reciprocal Rank (MRR)}, \emph{Recall@5}, to give a fair evaluation to
a search/ranking task.

\paragraph{Identity Query} --- This tasks takes x amounts of random recipes
and for each recipe, the ingredient list is inputted and the recipe should be
returned.
This task is used as a sanity check, as this tasks should in theory create the
same embeddings for any embedding models, such as the TF-IDF baseline, and thus
give those models 100\% accuracy.

\paragraph{LLM Processed Queries} --- This task takes x amounts of random
recipes, and for each recipe, inputs the ingredient list to a LLM with several
prompts to rephrase the input.
This is used to mimic a more human like input, while leveraging the power of
LLMs to quickly generate a larger dataset than e human annotated one.
For each recipe, the following prompts are used, and each promp is then used as
its own evaluation task:
\begin{itemize}
    \item \textbf{``Summarize this ingredient list into a short query a person might type
        into google''} --- This returns a more human-esque way of inputting a
        recipe, rather than a list of ingredients and instructions.
        Since text like this query, is supposed to be input into an LLM that
        then uses a processed input to a function, it was also asked to process
        this sentence into a more structure format containing a title and a list
        of ingredients.
    \item \textbf{``Extract only the main food items and their preparations from
        this list''} --- This returns a list that is more akin to the query a
        person (or program) is more likely to input to the model.
    \item \textbf{``I have the following ingredients and some of their
            preparations. List the most important ingredients and their
        preparations I should use to search a recipe with''} --- This returns a
        list that is compromised only of the most important elements of the
        recipe.
\end{itemize}

\paragraph{Human Processed Queries} --- This task takes y (smaller than x)
amounts of random recipes, and for each recipe, the author processed these list
into a more human like input.
For each list of ingredients, two different processing methods were used as
described below:
\begin{itemize}
    \item \textbf{A list of ingredients} --- for example ``Cucumber,
        sun-dried tomatoes, onion and ground beef'', mostly without preparations
        unless they are too core to the recipe, e.g.\ ``ground'' beef.
        This is to mimic a sample input of ``I have X food, what can I cook with it''
    \item \textbf{A list of prepared key ingredients} --- in this list the core
        and most unusal ingredients, with all of their preparations are compiled
        into a list; for example ``drained
        and quartered artichoke hearts; smoked turkey breast, cubed''.
        This is used to see if filtering on particular preparations can find
        very specific recipies.
\end{itemize}

\subsubsection{Query Speed}
To serve large langue models with a large amount of users, we need an efficient
system for querying our data, therefore query speed is an important factor in
deciding which method would be optimal for a specific use-case.

Since the models using only embeddings would require a cutoff point for
maximum number of results, different max number of results should be tested.

Query speeds are measured on all the tasks described in
Section~\ref{sec:modelperfomance}, as well as some custom query tasks
specifically constructed to test potentially slow queries, as described below:
\begin{itemize}
    \item \textbf{A very broad query} --- Something like ``salt'', and see how it
        handles a query that should find a lot of results.
    \item \textbf{A very specfic query} --- A very specific query that should
        return one or two recipies only.
    \item \textbf{A completely random query} --- A query of random ingredients,
        this should most likely not return anything at all.
\end{itemize}

\subsection{Ranking}
We need some way to perform ranking of the different models.
When using embeddings ranking gets trivial because we will use similarity
measures as our mean of searching either way.
A problem arise when we want to use normal database queries, as there is no way
to rank based of queries.
We can ofcourse sort by something like recipe rating, giving us the highest
rated recipes rather than the most relevant recipes (highest rated might be most
relevant).

An idea for how to rank the standar query version, is to use a score, similar to
tfidf for how many of the ingredients that is in the recipe, is in the query,
and negative score for any ingridient in the recipe that is not in the query.
For example if you have something like ``chopped tomato, parsley and tomato
paste'' as the query, you would add to the score for tomato, parsley and tomato
paste and any other ingredient would reduce the score.
The way something similar to tfidf can be used, is to decide how much the score
is affected by an ingredient, something like salt and pepper should not affect
the score too much, as these are prevalent in a lot of recipes and most likely
not something you would add to your query at all.
Using this, we could also assign scores to preparation steps, but this might be
redundant, as we will remove any recipe which does not contain the preparation
present in our query, and the point is you should be able to perform the
preparations if it is not already prepared.
In the above example, we would not find any recipes containing non-chopped
tomatoes, and if they are cooked as well, we can just cook our chopped tomatoes.

This does however raise the question, is a tfidf that also make use of our NER
tags to filter out results we do not want to see, sufficient? Maybe even more
efficient?
I also want to create a model using tfidf and filtering methods.
\subsection{NER Model}
\subsubsection{Labeling}

\section{Results}
\section{Discussion}
\section{Conclusion}

\end{document}

